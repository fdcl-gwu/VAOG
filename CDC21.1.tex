\documentclass[letterpaper, 10pt, conference]{ieeeconf}
\IEEEoverridecommandlockouts \overrideIEEEmargins
\usepackage{amsmath,amssymb,url}
\usepackage{graphicx}%,subfigure}
%\usepackage[all,import]{xy}
\usepackage{color}
\usepackage{siunitx}
%\usepackage[hidelinks]{hyperref}
\usepackage{cleveref}

\newcommand{\norm}[1]{\ensuremath{\left\| #1 \right\|}}
\newcommand{\abs}[1]{\ensuremath{\left| #1 \right|}}
\newcommand{\bracket}[1]{\ensuremath{\left[ #1 \right]}}
\newcommand{\braces}[1]{\ensuremath{\left\{ #1 \right\}}}
\newcommand{\parenth}[1]{\ensuremath{\left( #1 \right)}}
\newcommand{\ip}[1]{\ensuremath{\langle #1 \rangle}}
\newcommand{\refeqn}[1]{(\ref{eqn:#1})}
\newcommand{\reffig}[1]{Fig. \ref{fig:#1}}
\newcommand{\tr}[1]{\mbox{tr}\ensuremath{\negthickspace\bracket{#1}}}
\newcommand{\deriv}[2]{\ensuremath{\frac{\partial #1}{\partial #2}}}
\newcommand{\G}{\ensuremath{\mathsf{G}}}
\newcommand{\SO}{\ensuremath{\mathsf{SO(3)}}}
\newcommand{\T}{\ensuremath{\mathsf{T}}}
\renewcommand{\L}{\ensuremath{\mathsf{L}}}
\newcommand{\so}{\ensuremath{\mathfrak{so}(3)}}
\newcommand{\SE}{\ensuremath{\mathsf{SE(3)}}}
\newcommand{\se}{\ensuremath{\mathfrak{se}(3)}}
\renewcommand{\Re}{\ensuremath{\mathbb{R}}}
\newcommand{\Sph}{\ensuremath{\mathsf{S}}}
\newcommand{\aSE}[2]{\ensuremath{\begin{bmatrix}#1&#2\\0&1\end{bmatrix}}}
\newcommand{\ase}[2]{\ensuremath{\begin{bmatrix}#1&#2\\0&0\end{bmatrix}}}
\newcommand{\D}{\ensuremath{\mathbf{D}}}
\newcommand{\pair}[1]{\ensuremath{\left\langle #1 \right\rangle}}
\newcommand{\met}[1]{\ensuremath{\langle\!\langle #1 \rangle\!\rangle}}
\newcommand{\Ad}{\ensuremath{\mathrm{Ad}}}
\newcommand{\ad}{\ensuremath{\mathrm{ad}}}
\newcommand{\g}{\ensuremath{\mathfrak{g}}}
\DeclareMathOperator*{\argmin}{arg\,min}

\title{\LARGE \bf Variational Accelerated Optimization on a Lie Group}

\author{Taeyoung Lee%\authorrefmark{1}%
    \thanks{Taeyoung Lee, Mechanical and Aerospace Engineering, George Washington University, Washington, DC 10051 {\tt tylee@gwu.edu}}%
    %\thanks{\textsuperscript{\footnotesize\ensuremath{*}}This research has been supported in part by NSF under the grant CNS-1837382.}%
}


\newcommand{\RomanNumeralCaps}[1]{\textup{\uppercase\expandafter{\romannumeral#1}}}
\newcommand{\RI}{\RomanNumeralCaps{1}}
\newcommand{\RII}{\RomanNumeralCaps{2}}
\newcommand{\RIII}{\RomanNumeralCaps{3}}

\newcommand{\EditTL}[1]{{\color{red}\protect #1}}
\renewcommand{\EditTL}[1]{{\protect #1}}


\newtheorem{definition}{Definition}
\newtheorem{lem}{Lemma}
\newtheorem{prop}{Proposition}
\newtheorem{remark}{Remark}

\graphicspath{{./Figs/}}

%\color{white}\pagecolor{black}

\begin{document}
\allowdisplaybreaks


\maketitle \thispagestyle{empty} \pagestyle{empty}

\begin{abstract}
\end{abstract}

\section{Introduction}

\section{Extended Lagrangian Mechanics}

This section presents Lagrangian mechanics for non-autonomous systems on a Lie group. 
It is referred to as \textit{extended} Lagrangian mechanics as the variational principle is extended to include reparamerization of time~\cite{MarWesAN01}.
These are developed in both of continuous-time and discrete-time formulations, and the latter is often referred to as \textit{Lie group variational integrator}~\cite{LeeLeoCMAME07}.
They will be utilized in accelerated optimization with Bregman Lagrangian in the next section.

Consider  an $n$-dimensional Lie group $\G$.
Let $\g$ be the lie algebra, or the tangent space at the identity, i.e., $\g = \T_e\G$.
The tangent bundle of the group is identified with $\T\G \simeq G\times \g$ via left trivialization.
More specifically, let $\L:\G\times\G\rightarrow\G$ be the left action defined such that $\L_g h = gh$ for $g,h\in\G$.
For any $v\in\T_g\G$, there exists $\xi\in\g$ such that $v =\T_e\L_g \xi=g\xi$.
Utilizing this, the kinematics equation can be written as
\begin{align}
    \dot g = g\xi. \label{eqn:g_dot}
\end{align}
Further, suppose $\g$ is equipped with an inner product $\pair{\cdot, \cdot}$, which is extended to the corresponding inner product on $\T_g\G$ through the left trivialization, i.e., for any $v,w\in\T_g\G$, we have $\pair{w,v}_{\T_g\G} = \pair{ \T_g \L_{g^{-1}} v, \T_g \L_{g^{-1}} w}_\g$. 
Considering the inner product as a pairing between a tangent vector and a cotangent vector, $g\simeq g^*$ and $\T_g \G \simeq \T^*_g \G\simeq G\times \g^*$.
Throughout this paper, the pairing is also denoted by $\cdot$.
The adjoint operator is denoted by $\Ad_g:\g\rightarrow\g$, and the ad operator is denoted by $\ad_\xi:\g\rightarrow\g$. See, for example~\cite{MarRat99} for detailed preliminaries. 


\subsection{Continuous-Time Extended Lagrangian Mechanics}


Consider a non-autonomous Lagrangian $L(t,g,\xi):\Re\times\G\times\g\rightarrow \Re$ on the \textit{extended state space} including the space of time.
The corresponding \textit{extended path space} is composed of the curves $(c_t(a),c_g(a))$ on $\Re\times \G$ parameterized by $a>0$.
To ensure that the reparameterized time increases monotonically, we require $c'_t(a) > 0$. 
For a given interval of time $[t_0,t_f]$, the equivalent interval $[a_0,a_f]$ of $a$ is chosen such that $t_0=c_t(a_0)$ and $t_f=c_t(a_f)$.
For any path $(c_t(a),c_g(a))$ over $[a_0,a_f]$ in the extended space, the \textit{associated curve} $g$ is defined by
\begin{align}
    g(t) = c_g(c_t^{-1}(t)),\label{eqn:ac}
\end{align}
on $\G$ over $[t_0,t_f]$.
For a given extended path, define the \textit{extended action integral} as
\begin{align}
    \mathfrak{G}(c_t,c_g) = \int_{t_0}^{t_f} L(t,g,\xi)\bigg|_{g(t) = c_g(c_t^{-1}(t))} dt,\label{eqn:AI}
\end{align}
where the Lagrangian is evaluated through the associated curve \eqref{eqn:ac}.

Taking the variation of $\mathfrak{G}$ with respect to the extended path, we obtain the Euler-Lagrange equation according to the variational principle in the extended phase space. 
As discussed in~\cite[Sec. 4.2.2]{MarWesAN01}, the resulting Euler-Lagrangian equations depends only on the associated curve \eqref{eqn:ac}, not on the extended path $(c_t,c_g)$ itself, and the variational principle does not dictate how the curve should be reparameterized. 

Further, the resulting Euler-Lagrangian equation share the exactly same form as (unextended) Lagrangian mechanics for the associated curve. 
As such, the Euler-Lagrange equation for non-autonomous Lagrangian $L(t,g,\xi):\Re\times\G\times\g\rightarrow \Re$ can be written as
\begin{align}
    \frac{d}{dt}\!\parenth{\deriv{L}{\xi}} - \ad^*_\xi \deriv{L}{\xi} - \T^*_e \L_g (\D_g L) = 0,
\end{align}
where $\D_g$ stands for the differential with respect to $g$ (see~\cite[Sec. 8.6.3]{LeeLeo17} for derivation of the above equation for autonomous Lagrangian).

\subsection{Extended Lie Group Variational Integrator}

Variational integrators are geometric numerical integration schemes interpreted as discrete-time mechanics, and they are constructed by discretizing the variational principle for Lagrangian mechanics~\cite{MarWesAN01}.
The solution of variational integrators are symplectic and they preserve any symmetry according to the discrete analogue of Nother's theorem.
As such, they provide long-term structural stability in numerical simulation. 
For Lagrangian mechanics evolving on a Lie group, the corresponding Lie group variational integrators are presented in~\cite{Leo04,Lee08,LeeLeoCMAME07}.

Here, we present extended Lie group variational integrator by discretizing the extended variational principle presented above, by following the general framework of~\cite{MarWesAN01}.
The \textit{extended discrete path space} is composed of the sequence $\{(t_k, g_k)\}_{k=0}^N$ on $\Re\times\G$, satisfying $t_{k+1}>t_k$.
Next, the discrete kinematics equation is chosen as
\begin{align}
    g_{k+1} = g_k f_k, \label{eqn:gkp}
\end{align}
for $f_k \in\G$ representing the relative update of the state over a single timestep. 
The discrete Lagrangian $L_d(t_k, t_{k+1}, g_k, f_k): \Re\times\Re\times\G\times\G\rightarrow \Re$ is chosen such that the following \textit{extended discrete action sum}
\begin{align}
    \mathfrak{G}_d(\{(t_k, g_k)\}_{k=0}^N) = \sum_{k=0}^{N-1} L_d(t_k, t_{k+1}, g_k, f_k), \label{eqn:Gd}
\end{align}
approximates \eqref{eqn:AI}. 
Now the discrete Euler-Lagrange equations can be developed according to the variational principle as follows.

\begin{prop}
    The discrete path of $\{(g_k,f_k)\}_{k=0}^{N-1}$ extremizing \eqref{eqn:Gd} with fixed end point satisfies the following discrete Euler-Lagrange equation, or the extended Lie group variational integrator.
    \begin{gather}
        \T^*_e\L_{g_k}(\D_{g_k} L_{d_k})- \Ad^*_{f_k^{-1}} (\T^*_e\L_{f_k}(\D_{f_k} L_{d_k}))\nonumber \\
        + \T^*_e\L_{f_{k-1}}(\D_{f_{k-1}} L_{d_{k-1}}) =0,\label{eqn:DEL}\\
        \D_{t_k} L_{d_{k-1}} + \D_{t_k} L_{d_k} = 0, \label{eqn:DELt}
    \end{gather}
    along with \eqref{eqn:gkp}.
\end{prop}
\begin{proof}
    From \eqref{eqn:gkp},
    \begin{align*}
        \delta f_k = - g_k^{-1}( \delta g_k ) g_k^{-1} g_{k+1} + g_k^{-1}\delta g_{k+1}.
    \end{align*}
    Since $\delta g_k$ can be written as $\delta g_k = g_k \eta_k $ for $\eta_k\in \g$, 
    \begin{align*}
        \delta f_k = - \eta_k f_k +f_k \eta_{k+1}.
    \end{align*}
    Or equivalently, 
    \begin{align}
        f_k^{-1}\delta f_k = -\Ad_{f_k^{-1}} \eta_k + \eta_{k+1}.\label{eqn:del_fk}
    \end{align}

    Take the variation of \eqref{eqn:Gd} and substitute \eqref{eqn:del_fk} to obtain
    \begin{align*}
        \delta \mathfrak{G}_d  = \sum_{k=0}^{N-1}
        & \T^*_e\L_{g_k}(\D_{g_k} L_{d_k}) \cdot \eta_k \\
        & + \T^*_e\L_{f_k}(\D_{f_k} L_{d_k}) \cdot (-\Ad_{f_k^{-1}} \eta_k + \eta_{k+1}) \\
        & + \D_{t_k} L_{d_k}\cdot \delta t_k + \D_{t_{k+1}} \D_{d_k}\cdot \delta t_{k+1}.
    \end{align*}
    Since the endpoints are fixed, we have $\eta_0=0$ and $\delta t_0 = 0$.
    Therefore in the above expression, the range of summation for the terms paired with $\eta_k$ and $\delta t_k$ can be reduced to $1\leq k\leq N-1$. 
    Also, using $\eta_N=0$ and $\delta t_N=0$, for the other terms paired with $\eta_{k+1}$ and $\delta_{k+1}$, the subscript for the time index can be reduced by one for the same range of $k$.
    Therefore, 
    \begin{align*}
        \delta \mathfrak{G}_d  = \sum_{k=1}^{N-1}
        & \big\{ \T^*_e\L_{g_k}(\D_{g_k} L_{d_k})- \Ad^*_{f_k^{-1}} (\T^*_e\L_{f_k}(\D_{f_k} L_{d_k})) \\
        & + (\T^*_e\L_{f_{k-1}}(\D_{f_{k-1}} L_{d_{k-1}})) \big\} \cdot \eta_{k} \\
        & + \{ \D_{t_k} L_{d_k} + \D_{t_{k}} \D_{d_{k-1}} \} \delta t_k.
    \end{align*}
    According to the variational principle, $\delta\mathfrak{G}_d = 0$ for any $\eta_k$ and $\delta t_k$, which yields \eqref{eqn:DEL}.
\end{proof}
The most notable difference against the continuous-time counterpart is that in addition to the discrete Euler-Lagrange equation \eqref{eqn:DEL}, we have the additional equation \eqref{eqn:DELt} for the evolution of the discrete time. 
This is because the discrete action sum $\mathfrak{G}_d$ depends on the complete extended path $\{(t_k,g_k)\}_{k=1}^N$.
Whereas the action $\mathfrak{G}$ in the continuous time formulation is a function of the associated curve \eqref{eqn:ac}.
The evolution of the discrete time is associated with the energy.
Let the discrete energies be
\begin{align}
    E^+_k &= - D_{t_{k+1}} L_{d_k}\label{eqn:Ep}\\
    E^-_k &= D_{t_{k}} L_{d_k}.\label{eqn:Em}
\end{align}
Then, \eqref{eqn:DELt} can be rewritten as
\begin{align}
    E^+_{k-1} = E^-_k,\label{eqn:DELE}
\end{align}
which reflects to the evolution of the discrete energy.

In case the discrete Lagrangian is time-invariant so that $L_d(t_k,t_{k+1},g_k,f_k) =  L_d(t_k+s,t_{k+1}+s,g_k,f_k)$ for any $s$, 
we have $\D_{t_k}L_{d_k} + \D_{t_{k+1}}L_{d_k}=0$, or equivalently $E^-_k = E^+_k$. 
Combined with \eqref{eqn:DELE}, this yields $E_{k-1}^+ = E_k^+$ and $E_{k-1}^-=E_k^-$, which shows the conservation of discrete energy for autonomous Lagrangian systems. 

To implement \eqref{eqn:DEL} and \eqref{eqn:DELt}, it is more convenient to introduce the \textit{extended discrete Legendre transform}, $\mathbb{F}^\pm L_{d_k}: \Re\times\Re \times \G \times \G \rightarrow \Re\times \Re\times\G\times\g^*$
\begin{align}
    \mathbb{F}^+ L_{d_k} &= (t_{k+1}, E_{k+1}, g_{k+1}, \mu_{k+1}),\\
    \mathbb{F}^- L_{d_k} &= (t_k, E_k, g_{k}, \mu_{k}).
\end{align}
where
\begin{align}
    \mu_{k+1} & = \T^*_e\L_{f_k} \D_{f_k} L_{d_k},\label{eqn:mukp}\\
    \mu_k & = -\T^*_e\L_{g_k}(\D_{g_k} L_{d_k})+ \Ad^*_{f_k^{-1}} (\T^*_e\L_{f_k}(\D_{f_k} L_{d_k})),\label{eqn:muk}
\end{align}
and $E_{k+1}$ and $E_k$ are given by \eqref{eqn:Ep} and \eqref{eqn:Em}, respectively. 

The resulting discrete flow map is given by $\mathbb{F}^+L_{d_k} \circ (\mathbb{F}L_{d_k})^{-1}$. 
More specifically, for given $(t_k, E_k, g_k, \mu_k)$, \eqref{eqn:Em} and \eqref{eqn:muk} are solved together for $t_{k+1},f_k$ with the constraint $t_{k+1}>t_k$.
Then, $(E_{k+1}, g_{k+1},\mu_{k+1})$ are computed by \eqref{eqn:Ep}, \eqref{eqn:gkp}, and \eqref{eqn:mukp}, respectively.
This yields the discrete flow map $(t_k, E_k, g_k, \mu_k)\rightarrow(t_{k+1}, E_{k+1}, g_{k+1}, \mu_{k+1})$ consistent with \eqref{eqn:DEL} and \eqref{eqn:DELt}.

\section{Bregman Lagrangian Systems on $\G$}

On a second thought, do we really need Hamiltonian system, given that we have VI for the time-varying Lagrangian system. 

Bregman Lagrangian is
\begin{align*}
    L = e^{\alpha(t)+\gamma(t)} D_h(x+ e^{-\alpha(t)} v, x) - e^{\alpha+\beta+\gamma} f(x),
\end{align*}
where 
\begin{align*}
    \dot\beta \leq e^\alpha\\
    \dot \gamma = e^\alpha.
\end{align*}
In particular, the following choice has been made. 
For $p,C>0$,
\begin{align*}
    \alpha & = \log p - \log t\\
    \beta & = p\log t + \log C\\
    \gamma & = p \log t.
\end{align*}
Substituting these,
\begin{align*}
    L = p t^{p-1} D_h(x+ \frac{t}{p} v, x) - p C t^{2p-1} f(x).
\end{align*}
Further when $h(x)=\frac{1}{2}\|x\|^2$, 
\begin{align*}
    L = \frac{t^{p+1}}{2p} \| v\|^2 - p C t^{2p-1} f(x).
\end{align*}


\subsection{Continuous-Time EL}

Consider
\begin{align*}
    L(t,R,\Omega) = \frac{t^{p+1}}{2p} \Omega\cdot J\Omega - cpt^{2p-1} f(R).
\end{align*}

It has been shown that the variational principle in the extended space results in the same EL equation as for autonomous systems, and it does not matter how the time is reparameterized. 
We have
\begin{align*}
    \D_\Omega L = \frac{t^{p+1}}{p} J\Omega,\\
    \T^*_e \L_R( \D_R L) = cpt^{2p-1} M.
\end{align*}
and
\begin{align*}
    \frac{d}{dt} \D_\Omega L = \frac{p+1}{p} t^p J\Omega + \frac{t^{p+1}}{p} J\dot\Omega.
\end{align*}
Thus, the EL equation is given by
\begin{align*}
    \frac{t^{p+1}}{p} J\dot\Omega + \frac{p+1}{p} t^p J\Omega + \frac{t^{p+1}}{p} \hat\Omega J\Omega - cpt^{2p-1} M = 0,
\end{align*}
or
\begin{align*}
    J\dot\Omega + \frac{p+1}{t} J\Omega + \hat\Omega J\Omega - c p^2 t^{p-2} M = 0.
\end{align*}

Let
\begin{align*}
\Pi = \frac{t^{p+1}}{p} J\Omega
\end{align*}
We have
\begin{align*}
    \dot \Pi + \Omega\times \Pi - cpt^{2p-1} M = 0.
\end{align*}
and
\begin{align*}
    \dot R = \frac{p}{t^{p+1}} R (J^{-1}\Pi)^\wedge
\end{align*}


Let 
\begin{align*}
    f(R) & = \frac{1}{2}\| A- R\|^2_F = \frac{1}{2} \tr{(A-R)^T(A-R)}  \\
         & =\frac{1}{2}(\|A\|^2 + 3) - \tr{A^T R}.
\end{align*}
Thus,
\begin{align*}
    \D_R f(R) \cdot \delta R = -\tr{A^T R\hat\eta} = (A^T R- R^T A)^\vee \cdot \eta.
\end{align*}
This implies
\begin{align*}
    M = (R^T A-A^T R)^\vee.
\end{align*}

\subsection{LGVI Second Order Integrator}


The discrete Lagrangian is chosen as
\begin{align*}
    & L_d(t_k, t_{k+1}, R_k, F_k) = \frac{t_{k,k+1}^{p+1}}{h_k p} \tr{(I-F_k)J_d}\\
    &- \frac{h_k}{2}c p t_k^{2p-1} f(R_k) - \frac{h_k}{2}c p t_{k+1}^{2p-1} f(R_{k+1}),
\end{align*}
where $h_k = t_{k+1}-t_k$ and $t_{k,k+1}=(t_k + t_{k+1})/2$.

Let 
\begin{align*}
    M = -\T^*_I \L_R f(R).
\end{align*}
We have
\begin{align*}
    D_{R_k} L_{d_k} &= \frac{h_k}{2} cp (t_{k,k+1}^{2p-1} M_k + t_{k+1}^{2p-1} F_k M_{k+1})\\
    D_{F_k} L_{d_k} &= \frac{t^{p+1}_{k}}{h_k p} (J_dF_k -F_k^T J_d)^\vee + \frac{h_k}{2} cpt^{2p-1}_{k+1} M_{k+1} \\
    \Ad^*_{F_k^T} & \T^*_e \L_{F_{k+1}} D_{F_k} L_{d_k}\\
                    & = \frac{t^{p+1}_{k,k+1}}{h_k p} (F_k J_d - J_dF_k^T)^\vee + \frac{h_k}{2} cpt^{2p-1}_{k+1} F_k M_{k+1} \\
\end{align*}

Also, for $t_k$
\begin{align*}
    & D_{h_k} L_{d_k} = -\frac{t^{p+1}_{k,k+1}}{h_k^2 p} \tr{(I-F_k)J_d}\\
    & - \frac{1}{2}c p t_k^{2p-1} f(R_k) - \frac{1}{2}c p t_{k+1}^{2p-1} f(R_{k+1}),\\
    & D_{t_{k,k+1}} L_{d_k} = \frac{(p+1)t_{k,k+1}^p}{h_k p} \tr{(I-F_k)J_d}  
\end{align*}
Thus,
\begin{align*}
    & D_{t_k} L_{d_k} = \frac{1}{2} D_{t_{k,k+1}} L_{d_k} - D_{h_k} L_{d_k} \\
    & -\frac{h_k}{2} c p (2p-1) t_k^{2p-2} f(R_k) \\
    & = E^-_{d_k}.
\end{align*}
Next,
\begin{align*}
    & D_{t_{k+1}} L_{d_k} = \frac{1}{2} D_{t_{k,k+1}} L_{d_k} + D_{h_k} L_{d_k} \\
    & -\frac{h_k}{2} c p (2p-1) t_{k+1}^{2p-2} f(R_{k+1}) \\
    & = - E^+_{d_k}.
\end{align*}
We have
\begin{gather*}
    D_{t_k} L_{d_k} + D_{t_k} L_{d_{k-1}} = 0\\
    E^-_{d_k} = E^+_{d_{k-1}} \\
\end{gather*}

Perform discrete Legendre transform to obtain
\begin{gather*}
    \Pi_k =\frac{t_{k,k+1}^{p+1} }{h_k p} (F_k J_d - J_dF_k^T)^\vee  -\frac{h_k}{2} cp t_k^{2p-1} M_k \\
    \Pi_{k+1} = \frac{t_{k,k+1}^{p+1} }{h_k p} (J_dF_k -F_k^T J_d)^\vee + \frac{h_k}{2} cpt^{2p-1}_{k+1} M_{k+1}\\
= F_k^T \Pi_k + \frac{h_k}{2} cp t_k^{2p-1} F_k^T M_k +\frac{h_k}{2} cpt^{2p-1}_{k+1} M_{k+1}
\end{gather*}

First consider, $\hat{\mathbb{F}}^+ L_d: (t_k,t_{k+1},R_k,F_k)\rightarrow(t_{k+1}, R_{k+1}, \Pi_{k+1}, E_{k+1})$
\begin{align*}
&    \Pi_{k+1}  = F_k^T \Pi_k + \frac{h_k}{2} cp t_k^{2p-1} F_k^T M_k +\frac{h_k}{2} cpt^{2p-1}_{k+1} M_{k+1},\\
& E_{k+1} = - \frac{(p+1)t_{k,k+1}^p}{2 h_k p} \tr{(I-F_k)J_d}  \\
&     +\frac{t^{p+1}_{k,k+1}}{h_k^2 p} \tr{(I-F_k)J_d}\\
    & + \frac{1}{2}c p t_k^{2p-1} f(R_k) + \frac{1}{2}c p t_{k+1}^{2p-1} f(R_{k+1}),\\
    & +\frac{h_k}{2} c p (2p-1) t_{k+1}^{2p-2} f(R_{k+1}) \\
\end{align*}
Also, $\hat{\mathbb{F}}^- L_d: (t_k,t_{k+1}, R_k, F_k)\rightarrow (t_k, R_k, \Pi_k, E_k)$ given by
\begin{align*}
    \Pi_k & =\frac{t_{k,k+1}^{p+1}}{h_k p} (F_k J_d - J_dF_k^T)^\vee  -\frac{h_k}{2} cp t_k^{2p-1} M_k \\
    E_k & = \frac{(p+1)t_{k,k+1}^p}{ 2 h_k p} \tr{(I-F_k)J_d}  \\
    & +\frac{t^{p+1}_{k,k+1}}{h_k^2 p} \tr{(I-F_k)J_d}\\
    & + \frac{1}{2}c p t_k^{2p-1} f(R_k) + \frac{1}{2}c p t_{k+1}^{2p-1} f(R_{k+1}),\\
    & -\frac{h_k}{2} c p (2p-1) t_k^{2p-2} f(R_k) \\
\end{align*}
The discrete Hamiltonian flow map is constructed by $\hat{\mathbb{F}}^+L_d \circ (\hat{\mathbb{F}}^-L_d)^{-1}$.


\bibliography{/Users/tylee/Documents/BibMaster17,/Users/tylee/Documents/tylee}
\bibliographystyle{IEEEtran}

\end{document}

\begin{align*}
    H(R_k, P_{k+1}) = P_{k+1} \cdot R_k   - \frac{1}{h} \tr{ (I-F_k) J_d},
\end{align*}
where $F_k$ is the solution of \eqref{eqn:Pkp} for given $(R_k,F_k)$, which is rewritten as

The type II discrete Hamiltonian is
\begin{align*}
    H(R_k, P_{k+1}) = P_{k+1} \cdot R_k   - \frac{1}{h} \tr{ (I-F_k) J_d},
\end{align*}
where $F_k$ is the solution of \eqref{eqn:Pkp} for given $(R_k,F_k)$, which is rewritten as
\begin{align*}
    R_k^T P_{k+1} = \frac{1}{h}(F_k J_d F_k - J_d).
\end{align*}
Therefore, the Hamiltonian is rewritten in terms of $F_k$ into
\begin{align*}
    H(F_k(R_k, P_{k+1})) & = \frac{1}{h} \tr{(R_k^T P_{k+1})^T F_k} - \frac{1}{h}\tr{(I-F_k)J_d}\\
                         & = -\frac{1}{h} \tr{(I-F_k) J_d}.
\end{align*}
The variational of the discrete Hamiltonian is
\begin{align*}
    \delta H = \frac{1}{h}\tr{\delta F_k J_d}.
\end{align*}
We have to rewrite $\delta H$ in terms of $\delta R_k$ and $\delta P_{k+1}$.

The variation of \eqref{eqn:Pkp} is
\begin{align*}
    \delta  (R_{k+1}^TP_{k+1})  & = \frac{1}{h} (J_d F_k\hat \chi_k + \hat\chi_k F_k^T J_d) \\
                                & = \frac{1}{h} ((\tr{F^T J_d} I - F^T J_d) \chi_k)^\wedge.
\end{align*}
On the other hand, 
\begin{align*}
    \delta  (R_{k+1}^TP_{k+1}) & = \delta (F_k^T R_k^T P_{k+1}) \\
                               & = -\hat\chi_k R_{k+1}^T P_{k+1} - F_k^T \hat \eta_k R_k^T P_{k+1} + R_{k+1}^T \delta P_{k+1}
\end{align*}


